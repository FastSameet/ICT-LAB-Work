\documentclass[12pt]{article}

\usepackage{graphicx}
\usepackage{caption}
\usepackage{amsmath}
\usepackage{hyperref}
\usepackage{float}

\title{A Scientific Overview of the Horse}
\author{Sameet Siraj Ali}
\date{\today}

\begin{document}

\maketitle

\begin{abstract}
This paper provides a scientific overview of the horse, including its taxonomy, biological characteristics, diet, habitat, and scientific behavior. The document includes images, a data table, a citation to a scholarly article, and a hypothesis section with a mathematical lifespan model.
\end{abstract}

\section{Introduction}
The horse (\textit{Equus ferus caballus}) is a domesticated mammal known for its strength, speed, and long-standing partnership with humans. Horses play important roles in transportation, sport, therapy, agriculture, and cultural traditions throughout the world.

\section{Scientific Information}

\subsection{Classification}
Horses belong to the class Mammalia and the family Equidae. They are hoofed, herbivorous animals with highly developed muscle structure and cardiovascular capacity.

\subsection{Habitat}
Domesticated horses live worldwide, while wild horses inhabit grasslands, prairies, and open plains. They thrive in environments with abundant grazing land.

\subsection{Diet}
Horses are herbivores. Their diet primarily consists of grass, hay, grains, and plant stems. They require a high-fiber diet for proper digestion.

\section{Images}

(Replace these filenames with the names of the pictures you upload.)

\begin{figure}[H]
    \centering
    \includegraphics[width=0.7\textwidth]{fig1.png}
    \caption{A horse grazing in an open field.}
\end{figure}

\begin{figure}[H]
    \centering
    \includegraphics[width=0.7\textwidth]{fig2.png}
    \caption{Close-up image showing the horse's head and mane.}
\end{figure}

\begin{figure}[H]
    \centering
    \includegraphics[width=0.7\textwidth]{fig3.png}
    \caption{Horse running at high speed, demonstrating muscular build.}
\end{figure}

\begin{figure}[H]
    \centering
    \includegraphics[width=0.7\textwidth]{fig4.png}
    \caption{A group of horses in a natural habitat.}
\end{figure}

\section{Scientific Table}

\begin{table}[H]
\centering
\caption{Basic Scientific Information of the Horse}
\begin{tabular}{|c|c|}
\hline
\textbf{Detail} & \textbf{Information} \\
\hline
Scientific Name & \textit{Equus ferus caballus} \\
\hline
Class & Mammalia \\
\hline
Eats & Grass, hay, grains, plants \\
\hline
\end{tabular}
\end{table}

\section{Related Research}

According to the study in \cite{horseResearch2021}, horses exhibit advanced social behavior, strong memory retention, and complex communication through facial expressions and ear movement.

\section{Hypothesis About Horse Lifespan}

\subsection{Mathematical Lifespan Model}

We propose a simple formula to estimate the potential lifespan of a horse based on its average daily calorie intake.

Let:
\begin{itemize}
    \item $C$ = average daily calorie intake (kilocalories)
    \item $L$ = estimated lifespan (years)
\end{itemize}

Proposed equation:

\[
L = 0.002C + 15
\]

This model assumes that higher energy availability supports better long-term health, contributing to a longer lifespan. It is not a biological prediction but a mathematical hypothesis for academic purposes.

\section{Conclusion}
This paper presented scientific information about the horse, including taxonomy, diet, and behavior. A hypothesis was also proposed regarding their lifespan using a mathematical model. Horses remain one of the most influential domesticated animals in human history.

\bibliographystyle{plain}
\bibliography{references}

\end{document}
